
\subsection{Mitocondria Striatum}
For the Mitocondria Striatum set, CNN are employed to solve a segmentation problem. Every pixel is classified as being part of the mitocondria or not. A patch of 51x51 surrounding the pixel is extracted from the image and fed as input to a CNN which acts as a binary classifier. For determining the size and parameters of the CNN we used a set of
100.000 samples for training and 20.000 for validation (containing half positive and half negative samples). After determining the best setup, the network was trained on a larger set of 1 milion samples for training and 200.000 for validation.

The final test data represented a 3D volume consisting in 318 slices of 400x661 images. Thus, one frame contains 264.400 datasamples that are fed as test input for the CNN and all frames 97 million datasamples.

The best setup obtained using the small training and validation set was the one presented in Fig\ref{fig:CNN3}.
That one gave us a VOC error of on the first frame. We then trained this net on the big set and obtained an error of.

\begin{table}
\centering
\begin{tabular}{@{}rlll@{}}\toprule
Layer & Type & Maps and neurons& Kernel size \\ \midrule
0 & input & 1 map of 51x51 &\\
1& convolutional & 10 maps of 46x46 & 6x6\\
2 & max pooling & 10 maps of 23x23 &  \\
3 & convolutional & 20 maps of 18x18& 6x6 \\
4 & max pooling & 20 maps of 9x9& \\ 
3 & convolutional & 50 maps of 4x4& 6x6 \\
4 & max pooling & 50 maps of 2x2& \\ 
5 & fully conntected& 100 & \\
6 & fully conntected & 2 neurons & \\ \bottomrule
\end{tabular}
\caption{CNN for Mitocondria set}
\label{fig:CNN3}
\end{table}
